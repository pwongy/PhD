% Page style from here up to References
	\pagestyle{fancy}
	\lhead{{\sffamily \MakeUppercase\leftmark}}
	\chead{}
	\rhead{{\sffamily \MakeUppercase\rightmark}}
	\lfoot{}
	\cfoot{{\sffamily \thepage}}
	\rfoot{}
	
\chapter{Conclusions}

% Textbox
\begin{center}
	\begin{tcolorbox}[title=\boxtitle]
		\begin{itemize}[leftmargin=*,labelindent=2ex,labelsep=1.5ex,itemsep=0pt,parsep=0pt]
			\item What are the key findings of the thesis?
			\item How might the study be extended in the future?
		\end{itemize}
	\end{tcolorbox}
\end{center}

\section{Contributions of the Thesis}

To recap, the two main objectives of the thesis were: (i) to develop a realistic
computational model of the implanted cochlea for enabling knowledge development,
and (ii) to critically evaluate some of the assumptions currently used in volume
conduction models of the cochlea. Both of these aims have been achieved and the
key findings are summarised below.

Regarding the modelling process, the quality of the image data is key for
geometrical accuracy. This may be a challenge in clinical settings due to
restrictions in available imaging modalities. A high resolution volumetric
image stack is required for reconstructing the fine anatomy, such as the
cochlear membranes and vasculature, accurately. High resolution datasets are,
however, a double-edged sword in that the time and computational resources
needed to prepare the dataset and compute a solution to the system are
significant. Nevertheless, it is possible to do this using contemporary
technology.

There is currently no consensus as to which boundary condition assumption is
most appropriate. The studies here showed that they are not all equal. Modelling
the whole head would be preferred for accuracy but requires additional effort
and computation. For standalone models, the best boundary condition to impose
depends on the species of animal being modelled. In guinea pigs, the cochlea
protrudes into the tympanic bulla so current should exit via the temporal bone
only. In humans, a somewhat uniform current spread is expected, so grounding at
an infinitely large sphere may be an acceptable compromise. Simply grounding
these surfaces is not expected to yield close matches to measured intrascalar
voltages because the resistance of the true return path is not accounted for in
standalone models. This discrepancy should not be manipulated by changing the
resistivity of the surrounding bone domain because this can affect the predicted
current distribution. An alternative would be to model it using a voltage
offset, representing the effective voltage drop between the model boundary and
the monopolar return. This can be determined via comparisons with \invivo{}
data, and was found to have minimal effect on nerve fibre depolarisation as
predicted by the activating function (AF).

The accuracy of the material properties is another ongoing concern. Although
some of the resistivity values are well known, others have not been reliably
measured. For instance, some of the cochlear-specific tissue resistivities have
been converted from resistances that were fitted to a lumped-element model,
while others may only have a single data point in the literature. In addition,
there is no data on the permittivities of cochlear tissues, which are important
for determining the neural response more accurately.

Of the three models that were created in this thesis, the guinea pig model was
the most anatomically accurate. It was found to compare well with \invivo{} data
that was obtained independently at the Bionics Institute in Melbourne. As such,
this model was used for deeper investigations into two hitherto untested
modelling assumptions: the role of the cochlear vasculature in volume
conduction, and the validity of the quasi-static assumption.

Modelling the vascular structures required substantial additional effort due to
their pervasive and convoluted nature. The distribution of cochlear vessels was
clearly illustrated by Microfil-enhanced microCT scans, but even with high
resolution sTLIM images, it was not possible to reconstruct the entire vascular
tree. The vessels in the modiolus did have an effect on AF predictions, but
these were localised to the region nearest to the current source. The vein of
the scala tympani (VST) played a particularly noticeable role, so it is
recommended that future models include this structure as part of the
reconstruction.

It was possible to include time-dependent effects by using Fourier methods to
represent the current pulse and permittivity data for some of the tissues in the
cochlea. However, there is a very high computational cost due to the large
number of harmonics required for a clean signal, especially in simulations using
square-shaped stimuli. This can be significantly reduced by ramping the signal
at the leading and trailing edges of each phase. The quasi-static criterion did
not hold for nerve tissue across the modelled frequency range, so the assumption
of quasi-stationarity is not valid. Stationary analyses were shown to only
predict field values at the end of each phase. Time-dependent simulations, on
the other hand, demonstrated dynamic processes. For example, a relaxation effect
during the constant-current portions of the stimulus waveform contributed to an
asymmetric AF response between the cathodic and anodic phases, which may help to
explain some clinical observations on lead-phase performance differences.

In summary, the project has demonstrated that high fidelity computational
modelling of the electrophysiological response during CI stimulation is feasible
for research purposes. By creating this tool for \insilico{} investigations, our
understanding of the underlying relationships between the anatomy of the inner
ear, the current distribution through the tissues, and subsequent neural
response has been deepened. This in turn establishes a platform for further
studies, such as those in the following section.

\section{Future Research Directions}

\begin{verse}
	\textit{
		``By pointing out some of the gaps in our knowledge, I hope to motivate others
		to do some of the relevant experiments and theory.''}

	\vspace{4mm}

	\raggedleft{
		--- James Ranck, 1975~\cite{ranck1975}
	}
\end{verse}

There are a number of potential avenues for future investigations. First and
foremost, further validation of the field quantities should be undertaken to
verify that the model predictions are realistic throughout the cochlea, not only
in scala tympani. For instance, microelectrodes inserted into the
cochlea~\cite{thorne2004} may be able to provide a more complete electric field
map for comparison with the \insilico{} voltage distribution, and neural
response telemetry techniques~\cite{abbas1999,cohen2003} could be compared with
the predicted activating function along the neural sheet. This data should in
turn be used for additional testing of boundary conditions, including more
complex boundary conditions such as a distributed resistance to ground or a
whole head model (for the guinea pig). Further reducing uncertainties in this
manner will help to make the models more clinically relevant.

Secondly, additional measurements of the electric tissue properties should be
obtained using modern techniques~\cite{spelman1990}. These are one of the most
crucial inputs for the system but have yet to be determined reliably. The
resistivities of cochlear-specific tissues is particularly important as they
cannot be sourced from traditional compendiums. Both resistivity and
permittivity values across a wide frequency range should also be obtained to
enable further time-dependent simulations. There is substantial potential for a
new gold standard to be defined here.

Thirdly, the model may be used to investigate variations in electrode array
designs, intrascalar positioning, and other modes of stimulation. For the scope
of this thesis, a relatively standard array was inserted in a mid-scalar
position, and the focus was on improving simulations of monopolar mode. However,
the models can be adapted to allow for the virtual prototyping of more advanced
electrode array designs, to observe the effect of insertion trajectories that
are closer to or further from the modiolar wall, or to measure reductions in
effective current spread by switching to tripolar or phased-array stimulation.

Fourthly, there is scope remaining in this model to improve the nerve fibre
trajectories, the spacing of the nodes of Ranvier, and to include non-linear
kinetics at each node. Integration of these volume conduction models with more
advanced models of neural excitation would enable clinical outcomes such as
thresholds to be calculated. The prime candidates for this would be the GSEF
model~\cite{frijns1995} or the NEURON simulation environment~\cite{hines1997}.
As shown recently by Frijns~\etal, it is also possible to extend the modelling
even further downstream along the hearing pipeline once a comprehensive
excitation model has been coupled~\cite{frijns2015ciap}.

Lastly, the methodology used here has highlighted some significant remaining
challenges before patient-specific models can be generated easily. Clinical
imaging techniques do not provide sufficient resolution for high quality
reconstructions, so a level-of-detail study comparing, say, the guinea pig model
to less detailed versions would be useful for determining the error
differential. Even if high resolution scans become available in the future, the
accurate segmentation of such large data sets would require a large time
investment due to the lack of reliable automated algorithms. Geometry morphing
techniques based on a generic shape may provide a feasible
solution~\cite{hanekom2015ciap}. However, the method would need to account for
changes in the anatomy due to disease, such as the formation of scar tissue or
new bone and the degeneration of neural tissue. The discretisation step also
needs to be improved, as the current iteration, while robust, still requires
manual intervention at some stages. Different meshing programs might be sought,
or alternatively, the application of meshless methods could be
tested~\cite{horton2006}.

\section{Closing Remarks}

At the start of the twentieth century, the Wright brothers successfully
demonstrated that powered flight could be achieved. Over the following decades,
aircraft designs were iteratively and continually updated. The ability to fly
through the atmosphere revolutionised transport, and commercially viable
airlines facilitating intercontinental transport are now commonplace. The next
frontier---space---could not however be reached using those same designs because
the physics of flight did not transfer well to that environment.

\begin{verse}
	\textit{
		``We humans are wet, salty beasts, and we tend to conduct electricity pretty
		well. It not only goes where you want it to go, but also out into the
		surrounding tissues.''}

	\vspace{4mm}

	\raggedleft{
		--- Mark Bendett, 2012~\cite{boyle2012}}
\end{verse}

Cochlear implants (CIs) can be thought of as following a similar path.
Electrical stimulation of the cochlea has proven itself to be a successful
treatment for sensorineural hearing loss. However, evidence from multiple fronts
points toward the physics of the cochlea ultimately constraining the ability of
contemporary electrode array designs to deliver truly focused stimulation and
consistent patient outcomes. This suggests that electrical stimulation may not
be the best technique to use in the long run. In order to reach new heights in
hearing restoration, more revolutionary designs are required. Research into
alternative therapies is already underway, with early test results showing
promise in overcoming the limitations of electrical
stimulation~\cite{miller1997,izzo2006,izzo2007,richardson2009,jeschke2015}.
This forthcoming paradigm shift in CI designs will take many more years to reach
fruition\footnote{\href{https://xkcd.com/678/}{https://xkcd.com/678/}}, but no
matter which technology proves most viable, our understanding of the physical
relationships underpinning the system will be key to enabling improvements in
sound perception and quality of life for affected individuals.
